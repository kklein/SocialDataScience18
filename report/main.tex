%%%%%%%%%%%%%%%%%%%%%%%%%%%%%%%%%%%%%%%%%%%%%%%%%%%%%%%%%%%%%%%%%%%%%
% LaTeX Template: Project Titlepage Modified (v 0.1) by rcx
%
% Original Source: http://www.howtotex.com
% Date: February 2014
% 
% This is a title page template which be used for articles & reports.
% 
% This is the modified version of the original Latex template from
% aforementioned website.
% 
%%%%%%%%%%%%%%%%%%%%%%%%%%%%%%%%%%%%%%%%%%%%%%%%%%%%%%%%%%%%%%%%%%%%%%

\documentclass[12pt]{article}
\usepackage[a4paper]{geometry}
\usepackage[myheadings]{fullpage}
\usepackage{fancyhdr}
\usepackage{lastpage}
\usepackage{graphicx, wrapfig, subcaption, setspace, booktabs}
\usepackage[T1]{fontenc}
\usepackage[font=small, labelfont=bf]{caption}
\usepackage{fourier}
\usepackage[protrusion=true, expansion=true]{microtype}
\usepackage[english]{babel}
\usepackage{sectsty}
\usepackage{url, lipsum}

\newcommand{\HRule}[1]{\rule{\linewidth}{#1}}
\onehalfspacing
\setcounter{tocdepth}{5}
\setcounter{secnumdepth}{5}

%-------------------------------------------------------------------------------
% HEADER & FOOTER
%-------------------------------------------------------------------------------
\pagestyle{fancy}
\fancyhf{}
\setlength\headheight{15pt}
\fancyhead[L]{Kevin Klein}
%\fancyhead[R]{Anglia Ruskin University}
\fancyfoot[R]{Page \thepage\ of \pageref{LastPage}}
%-------------------------------------------------------------------------------
% TITLE PAGE
%-------------------------------------------------------------------------------

\begin{document}

\title{ \normalsize \textsc{Social Data Science project}
		\HRule{0.5pt} \\
		\LARGE How do internal and external reactions towards rises and falls of the S\&P 500 index differ?
		\HRule{2pt} \\ [0.5cm]
		\normalsize \today \vspace*{5\baselineskip}}

\date{}

\author{
		Kevin Klein \\ }

\maketitle
\tableofcontents
\newpage

%-------------------------------------------------------------------------------
% Section title formatting
\sectionfont{\scshape}
%-------------------------------------------------------------------------------

%-------------------------------------------------------------------------------
% BODY
%-------------------------------------------------------------------------------

\section{Motivation}
Humans are susceptible to a range of conscious biases [reference]. However, many times, one might look for an objective assessment of a situation. This project aims to address this desire in the realm of investing. Understanding the difference in perception of other people's and oneself's success,
in other words obtaining a more realistic perspective is a precondition to consciously and actively counteract biases such as the 'the grass is greener' effect.

? Assumptions
? Hypotheses
\section{Project plan}
\section{Data retrieval}
We refer to three datasets each comprising 254 weeks starting from February 17 2013. Each datapoint corresponds to a value assigned to one of the 254 weeks.

The financial data was provided by the Wall Street Journal \footnote{http://quotes.wsj.com/index/SPX/historical-prices} with daily opening and closing prices for the specified time range.

The Google query quantification has been provided by Google Trends. In particular the \emph{S\&P 500} market index topic could be used, instead of a simple keyword search. Unfortunately, for larger time ranges, results are restricted to be on a weekly basis. Moreover, this cannot be simply circumvented by merging data of many shorter time ranges, as the data represents locally normalized instead of global quantities. 

The reddit comment bodies stem from a publicly available dataset\footnote{https://files.pushshift.io/reddit/comments/} of all reddit comments from and including 2006 and 2017 . In particular, this dataset was access via Google Cloud's BigQuery platform \footnote{https://bigquery.cloud.google.com/dataset/fh-bigquery/reddit\_comments}. For a given week, all relevant comments were retrieved via a BigQuery library for python\footnote{https://cloud.google.com/bigquery/docs/reference/libraries\#client-libraries-install-python}. A comment was considered relevant if its body, turned lower case, contained the substring \emph{s\&p 500}. 

\section{Data processing}

? Normalization

For each date within the time range, the difference between closing and opening price was recorded and aggregated to a weekly mean, we will refer to this as \emph{change}. Weeks with strictly positive change are considered positive weeks, weeks with strictly negative change are considered negative weeks. Negative week changes were negated in order to represent positive values.

Per week, the number of relevant reddit comments as well as their average sentiment have been recorded. The sentiment analysis is based on nltk's SentimentIntensityAnalyzer\footnote{http://www.nltk.org/api/nltk.sentiment.html}. The latter provided a compound polarity score per datapoint.

\section{Analysis}

\section{Conclusion}

\section{Critique}

%-------------------------------------------------------------------------------
% REFERENCES
%-------------------------------------------------------------------------------
\newpage
\section*{References}
\addcontentsline{toc}{section}{References}

Anand, U., 2010. The Elusive Free Radicals, \textit{The Clinical Chemist,} [e-journal] Available at:<\url{http://www.clinchem.org/content/56/10/1649.full.pdf}> [Accessed 2 November 2013]
\newline
\newline

\end{document}

%-------------------------------------------------------------------------------
% SNIPPETS
%-------------------------------------------------------------------------------

%\begin{figure}[!ht]
%	\centering
%	\includegraphics[width=0.8\textwidth]{file_name}
%	\caption{}
%	\centering
%	\label{label:file_name}
%\end{figure}

%\begin{figure}[!ht]
%	\centering
%	\includegraphics[width=0.8\textwidth]{graph}
%	\caption{Blood pressure ranges and associated level of hypertension (American Heart Association, 2013).}
%	\centering
%	\label{label:graph}
%\end{figure}

%\begin{wrapfigure}{r}{0.30\textwidth}
%	\vspace{-40pt}
%	\begin{center}
%		\includegraphics[width=0.29\textwidth]{file_name}
%	\end{center}
%	\vspace{-20pt}
%	\caption{}
%	\label{label:file_name}
%\end{wrapfigure}

%\begin{wrapfigure}{r}{0.45\textwidth}
%	\begin{center}
%		\includegraphics[width=0.29\textwidth]{manometer}
%	\end{center}
%	\caption{Aneroid sphygmomanometer with stethoscope (Medicalexpo, 2012).}
%	\label{label:manometer}
%\end{wrapfigure}

%\begin{table}[!ht]\footnotesize
%	\centering
%	\begin{tabular}{cccccc}
%	\toprule
%	\multicolumn{2}{c} {Pearson's correlation test} & \multicolumn{4}{c} {Independent t-test} \\
%	\midrule	
%	\multicolumn{2}{c} {Gender} & \multicolumn{2}{c} {Activity level} & \multicolumn{2}{c} {Gender} \\
%	\midrule
%	Males & Females & 1st level & 6th level & Males & Females \\
%	\midrule
%	\multicolumn{2}{c} {BMI vs. SP} & \multicolumn{2}{c} {Systolic pressure} & \multicolumn{2}{c} {Systolic Pressure} \\
%	\multicolumn{2}{c} {BMI vs. DP} & \multicolumn{2}{c} {Diastolic pressure} & \multicolumn{2}{c} {Diastolic pressure} \\
%	\multicolumn{2}{c} {BMI vs. MAP} & \multicolumn{2}{c} {MAP} & \multicolumn{2}{c} {MAP} \\
%	\multicolumn{2}{c} {W:H ratio vs. SP} & \multicolumn{2}{c} {BMI} & \multicolumn{2}{c} {BMI} \\
%	\multicolumn{2}{c} {W:H ratio vs. DP} & \multicolumn{2}{c} {W:H ratio} & \multicolumn{2}{c} {W:H ratio} \\
%	\multicolumn{2}{c} {W:H ratio vs. MAP} & \multicolumn{2}{c} {\% Body fat} & \multicolumn{2}{c} {\% Body fat} \\
%	\multicolumn{2}{c} {} & \multicolumn{2}{c} {Height} & \multicolumn{2}{c} {Height} \\
%	\multicolumn{2}{c} {} & \multicolumn{2}{c} {Weight} & \multicolumn{2}{c} {Weight} \\
%	\multicolumn{2}{c} {} & \multicolumn{2}{c} {Heart rate} & \multicolumn{2}{c} {Heart rate} \\
%	\bottomrule
%	\end{tabular}
%	\caption{Parameters that were analysed and related statistical test performed for current study. BMI - body mass index; SP - systolic pressure; DP - diastolic pressure; MAP - mean arterial pressure; W:H ratio - waist to hip ratio.}
%	\label{label:tests}
%\end{table}